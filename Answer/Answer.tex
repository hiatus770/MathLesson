
\documentclass[10pt]{extarticle}
\usepackage{extsizes}
\usepackage{xcolor}
\usepackage{amsmath}
\usepackage{tikz}
\usepackage{graphicx}
\usepackage{fancyhdr}
\usepackage[paperwidth=210mm, paperheight=250mm, vmargin=2cm, hmargin=1.5cm]{geometry}

\pagestyle{fancy}
\fancyhf{}
\lhead{Answer Key}
\rhead{Page \thepage}
\renewcommand{\headrulewidth}{0.4pt} % Add this line

\makeatletter
\renewcommand{\maketitle}{
  \begin{center}
    {\Huge \@title}
    \vspace{0cm} % Adjust this value to change the space after the title
  \end{center}
}
\makeatother

\title{The Length of an Arc}
\begin{document}
\thispagestyle{empty}
\maketitle



\hrulefill


\section{Practice}
\textbf{Example 1.} Find the arc length of a unit semicircle (since we know the formula for the perimeter of a circle we know our answer should be $\frac{2\pi (1)}{2} = \pi$)
\begin{align}
    f(x) &= \sqrt{1-x^2} \\
    \text{Arc length} &= \int_{-1}^{1} \sqrt{1 + f'(x)^2} dx 
\end{align}

\begin{center}
\begin{tikzpicture}
    \draw[step = 0.5cm, gray, very thin] (-1.5,-1.5) grid (1.5,1.5);
    \draw (-1,0) arc (180:0:1cm);
\end{tikzpicture}
\end{center}

\textbf{SOLUTION: }
\begin{align}
    f'(x) &= \frac{1}{2\sqrt{1-x^2}} \cdot \frac{-2x}{1} = \frac{-x}{\sqrt{1-x^2}} dx \\ 
    L &= \int_{-1}^{1} \sqrt{1 + ({\frac{-x}{\sqrt{1-x^2}})^2}} dx \\ 
    L &= \int_{-1}^{1} \sqrt{1 + \frac{x^2}{1-x^2}} dx \\
    L &= \int_{-1}^{1} \frac{1}{\sqrt{1-x^2}} dx \\ 
    L &= \sin^{-1}(1) - \sin^{-1}(-1) \\ 
    L &= \pi 
\end{align}

\textbf{Example 2.} Find the arc length of $\ln(\sec(x))$ from $-\frac{\pi}{4}$ to $\frac{\pi}{4}$

\begin{align}
    f(x) &= \ln(\sec(x)) \\ 
    f'(x) &= \tan(x) \\ 
   \text{Arc Length} &= \int_{-\frac{\pi}{4}}^{\frac{\pi}{4}}  \sqrt{1 + (tan(x))^2} dx \\ 
    &= \int_{-\frac{\pi}{4}}^{\frac{\pi}{4}}  \sqrt{(\sec(x))^2} dx \\
    &= \int_{-\frac{\pi}{4}}^{\frac{\pi}{4}}  \sec(x) dx \\
    &= \ln(|\tan(\frac{\pi}{4}) + \sec(\frac{\pi}{4})|) -  \ln(|\tan(-\frac{\pi}{4}) + \sec(-\frac{\pi}{4})|) \\
    &= \ln(\sqrt{2}+1) - \ln(\sqrt{2} - 1)
\end{align}

\textbf{Example 3.} Find the arc length of $y = x^{3/2} $ from $x = 0$ to $x= 4$ 

\vspace{8cm}

\section{Finding lengths of parametric functions}

\textbf{Example 1.} Lets revisit our first example, finding the length of half an arc of a unit circle (from $0$ to $\pi$), since we can now rewrite it using the following parametric functions
\begin{align} 
    x = sin(t) \\ 
    y = cos(t) 
\end{align}

\textbf{SOLUTION: }

\begin{align}
    \frac{dx}{dt} &= \cos(x) \\ 
    \frac{dy}{dt} &= -\sin(x) \\
    L &= \int_{0}^{\pi} \sqrt{(\cos(x))^2 + (-\sin(x))^2} dx \\
    L &= \int_{0}^{\pi} 1 \cdot dx \\ 
    L &= \pi - 0 \\
    L &= \pi
\end{align}

\textbf{Example 2.} Find the length of the following parametric curve from $t = -2$ to $t = 2$

\begin{align} 
    x &= t^3 - 3t \\ 
    y &= 3t^2  
\end{align}


\begin{align} 
    \frac{dx}{dt} &= t^3 - 3t \\ 
    \frac{dy}{dt} &= 3t^2  
\end{align}

\section{"Difficulties"}
3. Find the arc length of an ellipse
\begin{align}
    f(x) &= \sqrt{1-\frac{x^2}{9}} \\
    f'(x) &= \frac{-\frac{2x}{9}}{2\sqrt{1-\frac{x^2}{9}}}
\end{align}

\begin{center}
    \begin{tikzpicture}
        \draw[step = 0.5cm, gray, very thin] (-3.5,-1) grid (3.5,3.5);
        \draw (-3,0) arc (180:0:3cm and 1cm); 
    \end{tikzpicture}
\end{center}

\end{document}